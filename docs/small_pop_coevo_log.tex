%Small_pop_coevo_log.tex
\documentclass [12pt, a4paper, twoside]  {article}
\usepackage [a4paper, left=30mm, right=30mm, top=20mm, bottom=20mm] {geometry}

%%% Paragraph formatting
\setlength{\parindent}{0pt} %new paragraph indent is 0
\setlength{\parskip}{1em}	%new paragraphs are indicated by a new line
\renewcommand{\baselinestretch}{1.2}

%%% Making sure line numbers are included
\RequirePackage{lineno}
\usepackage{lineno}

%%% Arranging graphics parameters
\usepackage [utf8] {inputenc}
\pagenumbering{gobble}

\title{
	\raggedright
	\Huge{\textbf{Small Population Coevolution Experiment}}\\
	\Large{Log, notes, and reminders}\\
	\Large{Jack Common}\\
	\vspace{.5cm}
}
\date{\vspace{-12ex}}

%% My own commands to make things a bit quicker to type
\newcommand{\super}{\textsuperscript}
\newcommand{\sub}{\textsubscript}

\begin{document}
\maketitle

\section*{Time-shift assay replication}
Identified which replicates in both treatments had phage until T5 and T7. 

1-clone:
\begin{itemize}
	\item T5 = 1, 3, 5, 7, 8, 9, 10, 11 ($N=8$)
	\item T7 = 1, 3, 5, 7, 8, 9, 10, 11 ($N=7$)
\end{itemize}

5-clone:
\begin{itemize}
	\item T5: 1, 2, 3, 4, 6, 7, 9, 10, 12 ($N=9$)
	\item T7: 2, 3, 6, 7, 9 ($N=5$)
\end{itemize}

Face a trade-off between sampling from more timepoints but with a smaller $N$, or increasing $N$ but sampling from fewer timepoints. Conducted a power analysis using the \texttt{pwr} package in R. Derived reference Cohen's $D$ values from Morley et al (2018) (small effect) and Hall et al (2011) (large effect). 1-clone, small ES (0.57): $D\sub{1,6}=0.43$; 1-clone, large ES (1.28): $D\sub{1,6}=0.76$. 5-clone, small ES (0.57): $D\sub{1,4}=0.30$; 5-clone, large ES (1.28): $D\sub{1,4}=0.56$.
The smaller sample sizes are therefore appropriate, but bear in mind this only relates to within-treatment comparisons across timepoints. The unbalanced $N$ between treatments will have to be taken into account when making inferences about treatment effect. 

\section*{Plaques in time shift experiment}
Noticed that when hosts from T3 of replicate 10 in the 1-clone treatment were challenged against phage, phage from T3 and T5 formed vague plaques, but phage from T7 formed very large and very clear plaques (at least twice the diameter of T3 and 5 phage)

\end{document}
